% ;; TeX-PDF-mode: t; TeX-master: "main" -*-%

%\documentclass[article,dr=phil,type=drfinal,colorback,accentcolor=tud9c]{tudthesis}
\documentclass[]{llncs}
%\usepackage{ngerman}
\overfullrule=1mm
\usepackage{epigraph}
\usepackage{xspace}
\usepackage[toc]{glossaries}
\usepackage{import}
%\usepackage{enumitem}      % adjust spacing in enums
\usepackage[colorlinks=true,allcolors=blue,breaklinks,draft=false]{hyperref}   % hyperlinks, includ
%\usepackage{ctable}
\usepackage{footmisc}
\usepackage{hhline}
\usepackage{etoolbox}
%\usepackage{amssymb}
%\usepackage{amsmath}
%\usepackage{mathtools}
%\usepackage{bm}

\usepackage{pgfplots}
\usepackage{booktabs}
%\usepackage{amsthm}
\usepackage{colortbl}
\usepackage{color}
%\usepackage{mparhack}
%\usepackage{pdfcomment}
\usepackage[draft]{fixme}
%\fxloadlayouts{pdfmargin}
%\fxsetup{pdfmargin}
%\fxsetup{margin}
\usepackage{multicol}
\usepackage{multirow}
\usepackage{url}

\usepackage{listings}
\lstset{
  frame=Ltb,
  framerule=0pt,
  aboveskip=0mm,
  framextopmargin=0pt,
  framexbottommargin=0pt,
  framexleftmargin=0cm,
  framesep=0pt,
  rulesep=0pt,
  showstringspaces = false,
  basicstyle=\ttfamily,
  numbers=left,
  mathescape=true,
  numbersep=-5pt,
  numberstyle=\tiny,
  numberfirstline=true,
  firstnumber=auto,
  breaklines=true
}

\usepackage{tikz}
\usetikzlibrary{shapes}
\usetikzlibrary{arrows}
\usetikzlibrary{decorations.markings}
\usetikzlibrary{shapes.multipart}
\usetikzlibrary{shapes.arrows}
\usetikzlibrary{shapes.geometric}
\usetikzlibrary{positioning}
\usetikzlibrary{patterns}
\usetikzlibrary{calc,decorations.pathreplacing,matrix,shadows,fit,chains} 
\usetikzlibrary{trees,hobby,backgrounds}
\pgfplotsset{compat=1.13}
%\usepackage{algorithmicx}
%\usepackage[noend]{algpseudocode}
%\algrenewcommand\algorithmicindent{4pt}
%\renewcommand{\algorithmicforall}{\textbf{for each}}
%\usepackage{algorithm}
%\algrenewcommand\algorithmicindent{2.0em}


%\DeclarePairedDelimiter\nat{\|}{\|}
%\DeclarePairedDelimiter\tuple{\langle}{\rangle}
%\DeclarePairedDelimiter\paren{(}{)}
%\DeclarePairedDelimiter\set{\{}{\}}
%\DeclarePairedDelimiter\multiset{\{\!\!\{}{\}\!\!\}}



\author{Antonio Flores Montoya}
\title{Datalog Disassembler}
\begin{document}
\maketitle

\section{Instruction boundaires}
-traverse code, considering possible starting points any address found in the code.

-discard invalid instructions and instruction that overlap with the traversed
instructions

-whenever a block ends try to disassembly contiguous instructions

At the end of the process we have a set of blocks. Some of them might be overlaping.
Use heuristics to resolve conflics
based on predecessors that do not overlap, symbols, if their address appears in the code,
etc.


\section{Symbolization}
\subsection{Auxiliary analyses}
Used Defined analysis:

compute dependencies backward of registers that are (transitively) used to access
memory.


Register Value analysis

Using the computed dependencies, compute a limited version of value analysis with a
very specific format forward.
Edges are added for operations that can be abstracted to the representaiton
the transitive relations are computed afterwards.

for binary operations the semantics are encoded in the second relation.



\subsection{Data access patterns}
\begin{itemize}
\item Using the results of the value analysis compute addresses that are accessed with
a certain size and multiplier.
\item Accesses within the range of the multiplier and with the same multiplier are considered
paired. (the access the same array of structs).
\item Data accesses are propagated until a new access is found.
If two accesses coincide, the one originated closer takes precendence and the
other one is not propagated further.
paired accesses are only propagated insofar as all of the do not reach other access.
\end{itemize}
\subsection{Other heuristics}

address array
aligned address

\subsection{Final decision}
An address is considered a symbol if it qualifies for at least two heuristics.


\section{Future work}

  Some possible next steps are:
  \begin{enumerate}
  \item sprinkle nops randomly in the code:
    Done, see Sect.\ref{sec:experiments}.
  \item run with compiled souffle:
    Done, see Sect.\ref{sec:experiments}.
  \item run against binaries with the optimize switch "dense computed jumps"
    I can just keep looking though fallthrough and should be fine

  \item Add a data delineation analysis:
    I'm getting a bunch of info about it but only for static memory accesses
    Explicitly avoiding checking stack accesses and not recording heap accesses.
    
    I read the paper and the objectives are quite different but maybe the analyses
    can be retargeted. The paper about data delineation is concerned with object
    passed as parameters.

  \item Add a function boundary analysis:

    I have some basic heuristics in place, but I would need DVT integration
    to check how well it works.

  \item DVT support for hints:
    Maybe get function dvt information
  \item Use TSL or QFBV
    Question about TSL. It is very big because every operation is split according to
    the sizes? Is there a way to get QFBV, is it worth it?

    The current analysis focuses on the address computation. This means a limited number
    of instructions are considered (only integer instructions)
    For the code inference, only jumps, calls and rets are relevant.
    That means that the relevant instructions are not so many, the effects of these could
    be factored out of the analysis easily (some inline relations and magic set)

  \end{enumerate}


\section{Transitions and associated risks}
\label{sec:risks}
Some possible transitions are:
\begin{enumerate}
\item  A playground for testing new heuristics
\item  A source of analysis contributing to our "hints database"
\item  A replacement binary front end for our flagship CodeSonar commercial tool
\item  A disassembler targeting our binary rewriting IR (in development)
\end{enumerate}

\subsection{Risks}

\paragraph{Dead code}
I realize that with binary stripping, functions are skipped but it does not seems to
affect functionality. They are probably dead code (procceeding from libraries)
but depending on the application we might want to obtain them.

\paragraph{Stack usage and heap usage}
Right now the dissassembler ignores all stack and heap memory accesses because for
reassembleabilty only static data is important.
However, for some applications this might change.

\paragraph{Architecture and compiler dependency}
As it stands, the disassembler works only for x64 architecture. There
are aspects of the disassembler that might require a considerable
effort to port to other architectures.

\begin{itemize}
\item For All the code inference algorithm assumes code
  is in .text and there is no data in .text which does not hold for ARM and apparently
  does not always hold for x86/windows.

\item 
  In x64, relative addressing is done using the register RIP, but this
  cannot be done in x86 and might be different in ARM.

\item
  The size of a pointer in x64 is 8 bits which is not a commonly used
  primitive data size. In 32 bits machines, using heuristics based on the
  size of the data type might be less effective.

\end{itemize}

Reasons (According to Tom Johnson) why x86/windows (older code like
90's, early 2000's) might be more challenging:
\begin{itemize}
\item Standard libraries that came w/ Visual Studio had what seemed to
  be hand-optimized code - particularly in the math library. Also, I
  remember printf having data tables in the code section.
\item Predates ASLR, so no need to keep relocation information in the
  executable.
\item No RIP-relative addressing. 
\end{itemize}

\section{Experiments}
\label{sec:experiments}
  \begin{itemize}
    \item
   Three compilers (versions): gcc 5.4, gcc 8, and clang 3.8
\item
  Five optimization flags: None, -O1, -O2, -O3 and -Os
  \item
    Three benchmarks: Coreutils, (a subset of) CGC, Real world examples
  \end{itemize}
  The list of real world examples is in Table\ref{table:real_examples}.
  \begin{table}
    \caption{Real world examples}
    \label{table:real_examples}
    \begin{tabular}{llllll}
      grep-2.5.4 &
      gzip-1.2.4  &
      bar-1.11.0  &
      conflict-6.0 &
      ed-0.2 &
      ed-0.9 \\ 
      marst-2.4 &
      units-1.85 &
      doschk-1.1 & 
      bool-0.2 &
      m4-1.4.4 &
      patch-2.6.1 \\ 
      enscript-1.6.1 &
      bison-2.1 &
      sed-4.2 & 
      flex-2.5.4 &
      make-3.80 & 
      rsync-3.0.7 \\
      gperf-3.0.3 & 
      re2c-0.13.5 & 
      lighttpd-1.4.18 &
      lighttpd-1.4.11 &
      tar-1.29 &
    \end{tabular}
  \end{table}

  \begin{table}
\caption{Funtionality tests}
    \begin{tabular}{llllll}
    Benchmark       & Examples    & Combinations   & Succeed   & Failed  & \%\\ \hline
    Coreutils       & 106         & 1590           & 1590      &   0     & 100\% \\
    CGC             & 86          & 1290           & 1268      & 18      & 98.2\% \\
    Real world & 23 & 345 & 342 & 3 & 99\% \\
  \end{tabular}
\end{table}
  The binaries are stripped (with the flag \texttt{-strip-unneeded})
  before rewriting. Once the assembler code has been obtained, the
  result has is stirred as follows: for each instruction, between 1
  and 10 nop instructions are added randomly with 1/3 probability
  throughout the code.


  In many of the examples that fail, the original binary produces
  different test results in different executions. So it is quite possible that
  the binaries are not broken.

  \subsection{Performance}
  Fig.\ref{fig:perf} contains the performance results of the analysis
  for the different benchmarks and the relation between analysis time
  and the size of the binaries.
  
  \begin{figure}
    \begin{tabular}{@{}l@{}l}
   \multicolumn{2}{l}{Performance of Datalog for coreutils}\\
      \begin{tikzpicture}
        \begin{axis}[
            width=0.50\textwidth,
            xlabel=Size (bytes),
            ymax=20,
            ylabel=Datalog time (seconds)]
          \addplot+[
            only marks,
            scatter,
            mark size=2pt]
          table[x index=3,y index=2]
          {stats_coreutils.csv};
        \end{axis}
      \end{tikzpicture}
      &
      \begin{tikzpicture}
        \begin{axis}[
             width=0.50\textwidth,
             xlabel=Size (bytes),
             ymax=20,
            ylabel=Total time (seconds)]
          \addplot+[
            only marks,
            scatter,
            mark size=2pt]
          table[x index=3,y index=0]
          {stats_coreutils.csv};
        \end{axis}
      \end{tikzpicture}\\
      Performance in CGC examples\\
      
        \begin{tikzpicture}
          \begin{axis}[
              width=0.50\textwidth,
              xlabel=Size (bytes),
              ymax=40,
              ylabel=Datalog time (seconds)]
            \addplot+[
              only marks,
              scatter,
              %  mark=halfcircle*,
              mark size=2pt]
            table[x index=3,y index=2]
            {stats_CGC.csv};
          \end{axis}
        \end{tikzpicture}
        &
        \begin{tikzpicture}
          \begin{axis}[
              width=0.50\textwidth,
              xlabel=Size (bytes),
              ymax=40,
              ylabel=Total time (seconds)]
            \addplot+[
              only marks,
              scatter,
              %  mark=halfcircle*,
              mark size=2pt]
            table[x index=3,y index=0]
            {stats_CGC.csv};
          \end{axis}
        \end{tikzpicture}\\
Performance in real world examples\\
     \begin{tikzpicture}
       \begin{axis}[
           width=0.50\textwidth,
           xlabel=Size (bytes),
           ymax=55,
          ylabel=Datalog time (seconds)]
        \addplot+[
          only marks,
          scatter,
          mark size=2pt]
        table[x index=3,y index=2]
        {stats_real.csv};
      \end{axis}
    \end{tikzpicture}
    &
    \begin{tikzpicture}
      \begin{axis}[
          width=0.50\textwidth,
          xlabel=Size (bytes),
          ymax=55,
          ylabel=Total time (seconds)]
        \addplot+[
          only marks,
          scatter,   
          mark size=2pt]
        table[x index=3,y index=0]
        {stats_real.csv};
      \end{axis}
    \end{tikzpicture}
  \end{tabular}
    \caption{Performance graphs. From top to bottom: Coreutils, CGC,
      and real world examples.  The datalog processing time on the
      left-hand side and the total processing time on the right-hand
      side.}
    \label{fig:perf}
\end{figure}

\bibliographystyle{plain}

\bibliography{biblio}

\end{document}
